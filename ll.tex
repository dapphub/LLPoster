%%%%%%%%%%%%%%%%%%%%%%%%%%%%%%%%%%%%%%%%%
% a0poster Portrait Poster
% LaTeX Template
% Version 1.0 (22/06/13)
%
% The a0poster class was created by:
% Gerlinde Kettl and Matthias Weiser (tex@kettl.de)
%
% This template has been downloaded from:
% http://www.LaTeXTemplates.com
%
% License:
% CC BY-NC-SA 3.0 (http://creativecommons.org/licenses/by-nc-sa/3.0/)
%
%%%%%%%%%%%%%%%%%%%%%%%%%%%%%%%%%%%%%%%%%

%----------------------------------------------------------------------------------------
%	PACKAGES AND OTHER DOCUMENT CONFIGURATIONS
%----------------------------------------------------------------------------------------

\documentclass[a0,portrait]{a0poster}

\usepackage{multicol} % This is so we can have multiple columns of text side-by-side
\columnsep=100pt % This is the amount of white space between the columns in the poster
\columnseprule=3pt % This is the thickness of the black line between the columns in the poster

\usepackage[svgnames]{xcolor} % Specify colors by their 'svgnames', for a full list of all colors available see here: http://www.latextemplates.com/svgnames-colors

\usepackage{times} % Use the times font
%\usepackage{palatino} % Uncomment to use the Palatino font

\usepackage{graphicx} % Required for including images
\graphicspath{{figures/}} % Location of the graphics files
\usepackage{booktabs} % Top and bottom rules for table
\usepackage[font=small,labelfont=bf]{caption} % Required for specifying captions to tables and figures
\usepackage{amsfonts, amsmath, amsthm, amssymb} % For math fonts, symbols and environments
\usepackage{prftree}
\usepackage{txfonts}
\usepackage{tabularx, array}
\usepackage{wrapfig} % Allows wrapping text around tables and figures

\newcolumntype{C}{>{\centering\arraybackslash$}X<{$}}
\newcommand{\parr}{\rotatebox[origin=c]{180}{$\&$}}

\begin{document}

%----------------------------------------------------------------------------------------
%	POSTER HEADER
%----------------------------------------------------------------------------------------

% The header is divided into two boxes:
% The first is 75% wide and houses the title, subtitle, names, university/organization and contact information
% The second is 25% wide and houses a logo for your university/organization or a photo of you
% The widths of these boxes can be easily edited to accommodate your content as you see fit

\begin{minipage}[b]{0.75\linewidth}
\veryHuge \color{NavyBlue} \textbf{blockchain linear logic cheat sheet} \color{Black}\\ % Title
\Huge\textit{much logic, such formal, wow}\\[2cm] % Subtitle
\huge \textbf{Denis Erfurt\& Jack Ek}\\[0.5cm] % Author(s)
\huge DappHub\\[0.4cm] % University/organization
\Large \texttt{denis@dapphub.com}\\

\end{minipage}
%
\begin{minipage}[b]{0.25\linewidth}
\includegraphics[width=20cm]{logo.jpg}\\
\end{minipage}

\vspace{1cm} % A bit of extra whitespace between the header and poster content

%----------------------------------------------------------------------------------------

\begin{multicols}{2} % This is how many columns your poster will be broken into, a portrait poster is generally split into 2 columns

%----------------------------------------------------------------------------------------
%	ABSTRACT
%----------------------------------------------------------------------------------------

% \color{Navy} % Navy color for the abstract

\begin{abstract}


\end{abstract}

%----------------------------------------------------------------------------------------
%	INTRODUCTION
%----------------------------------------------------------------------------------------

% \color{DarkSlateGray}

\section*{LL Sequent Calculus}

\begin{tabularx}{\linewidth}{CCC}
  \prftree[r]{?}
  {\vdash \Gamma, A}
  {\vdash \Gamma, ?A}
  &
  \prftree[r]{w}
  {\vdash \Gamma}
  {\vdash \Gamma, ?A}
  &
  \prftree[r]{c}
  {\vdash \Gamma, ?A, ?A}
  {\vdash \Gamma, ?A}
  \\ \\
  \prftree[r]{!}
  {\vdash ?A_1,...,?A_n, B}
  {\vdash ?A_1,...,?A_n, !B}
  &
  \prftree[r]{\&}
  {\vdash \Gamma, A}
  {\vdash \Gamma, B}
  {\vdash \Gamma, A\& B}
  &
  \prftree[r]{$\oplus_1$}
  {\vdash \Gamma, A}
  {\vdash \Gamma, A\oplus B}
  \\ \\
  \prftree[r]{$\oplus_2$}
  {\vdash \Gamma, B}
  {\vdash \Gamma, A\oplus B}
  &
  \prftree[r]{$\otimes$}
  {\vdash \Gamma, A}
  {\vdash B\Delta}
  {\vdash \Gamma, A\otimes B\Delta}
  &
  \prftree[r]{$\parr$}
  {\vdash \Gamma, A, B}
  {\vdash \Gamma, A\parr B}
\end{tabularx}

\section*{ILL Sequent Calculus}

\begin{tabularx}{\linewidth}{CC}
  \prftree[r]{$\otimes$L}
  {}
  {}
  {}
  &
  \prftree[r]{$\otimes$R}
  {}
  {}
  {}
  \\

  \prftree[r]{$\oplus$L}
  {\Delta, A\vdash C}
  {\Delta, B\vdash C}
  {\Delta, A\oplus B\vdash C}
  &
  \prftree[r]{$\oplus R_i$}
  {\Delta\vdash A_i}
  {\Delta\vdash A_0\oplus A_1}
\end{tabularx}


\section*{Negation and De-Morgan's laws}

\begin{tabularx}{\linewidth}{CCC}
  (A^\bot)^\bot \equiv A &
  (A\otimes B)^\bot \equiv A^\bot \parr B^\bot &
  (A\& B)^\bot \equiv A^\bot \oplus B^\bot \\ \\
  (A\parr B)^\bot \equiv A^\bot \otimes B^\bot &
  (A\oplus B)^\bot \equiv A^\bot \& B^\bot &
\end{tabularx}

\section*{something}

\begin{center}
  \begin{tabular}{r|cc}
    &  conjunction & disjunction \\
    multiplicative & $\otimes$ & $\parr$ \\
    additive & $\&$ & $\oplus$
  \end{tabular}
\end{center}

\begin{center}
  \begin{tabular}{rcc}
    positive &:& $\otimes$, $\oplus$, $\exists$ \\
    negative &:& $\&$, $\parr$, $\forall$
  \end{tabular}
\end{center}


\section*{Structural Rules}

\subsubsection*{distributivity}

\begin{tabularx}{\linewidth}{CC}
  A \otimes (B\oplus C) \multimapboth (A\otimes B) \oplus (A \otimes C)
  &
  A \parr (B\& C) \multimapboth (A\parr B) \& (A \parr C)
\end{tabularx}




\subsubsection*{RuleZer}

\begin{tabularx}{\linewidth}{CCC}
  \prftree[r]{$Id$}{}{   ? A      \vdash   ? A      }
&
  \prftree[r]{$Prem$}{}{ \Gamma  \vdash \Delta  }
&
  \prftree[r]{$Partial$}{}{ \Gamma  \vdash \Delta  }
\end{tabularx}


\subsubsection*{RuleCut}

\begin{tabularx}{\linewidth}{CCC}
  \prftree[r]{$Cut$}{  \Theta  ,  A     \vdash \Delta  }{ \Gamma  \vdash  A    }{  \Gamma  , \Theta   \vdash \Delta  }
\end{tabularx}


\subsubsection*{RuleStruct}

\begin{tabularx}{\linewidth}{CCC}
  \prftree[r]{$PL$}{  (  \Gamma_1  , \Delta_1   ) , (  \Gamma_2  , \Delta_2   )  \vdash \Theta  }{  (  \Gamma_1  , \Gamma_2   ) , (  \Delta_1  , \Delta_2   )  \vdash \Theta  }
&
  \prftree[r]{$PR$}{ \Theta  \vdash  (  \Delta_1  , \Gamma_1   ) , (  \Delta_2  , \Gamma_2   )  }{ \Theta  \vdash  (  \Delta_1  , \Delta_2   ) , (  \Gamma_1  , \Gamma_2   )  }
&
  \prftree[r]{$AR$}{ \Gamma  \vdash  (  \Delta_1  , \Delta_2   ) , \Delta_3   }{ \Gamma  \vdash  \Delta_1  , (  \Delta_2  , \Delta_3   )  }
\\\\
  \prftree[r]{$AR$}{ \Gamma  \vdash  \Delta_1  , (  \Delta_2  , \Delta_3   )  }{ \Gamma  \vdash  (  \Delta_1  , \Delta_2   ) , \Delta_3   }
&
  \prftree[r]{$AL$}{  (  \Delta_1  , \Delta_2   ) , \Delta_3   \vdash \Gamma  }{  \Delta_1  , (  \Delta_2  , \Delta_3   )  \vdash \Gamma  }
&
  \prftree[r]{$AL$}{  \Delta_1  , (  \Delta_2  , \Delta_3   )  \vdash \Gamma  }{  (  \Delta_1  , \Delta_2   ) , \Delta_3   \vdash \Gamma  }
\\\\
  \prftree[r]{$IL_L$}{   \cdot   , \Gamma   \vdash \Delta  }{ \Gamma  \vdash \Delta  }
&
  \prftree[r]{$IL_L$}{ \Gamma  \vdash \Delta  }{   \cdot   , \Gamma   \vdash \Delta  }
&
  \prftree[r]{$IL_R$}{  \Gamma  ,  \cdot    \vdash \Delta  }{ \Gamma  \vdash \Delta  }
\\\\
  \prftree[r]{$IL_R$}{ \Gamma  \vdash \Delta  }{  \Gamma  ,  \cdot    \vdash \Delta  }
&
  \prftree[r]{$IR_L$}{ \Gamma  \vdash   \cdot   , \Delta   }{ \Gamma  \vdash \Delta  }
&
  \prftree[r]{$IR_L$}{ \Gamma  \vdash \Delta  }{ \Gamma  \vdash   \cdot   , \Delta   }
\\\\
  \prftree[r]{$IR_R$}{ \Gamma  \vdash  \Delta  ,  \cdot    }{ \Gamma  \vdash \Delta  }
&
  \prftree[r]{$IR_R$}{ \Gamma  \vdash \Delta  }{ \Gamma  \vdash  \Delta  ,  \cdot    }
\end{tabularx}


\subsubsection*{RuleU}

\begin{tabularx}{\linewidth}{CCC}
  \prftree[r]{$\otimes_L$}{  \Gamma  , (   A    ,  B     )  \vdash  C    }{  \Gamma  ,   A  \otimes B      \vdash  C    }
&
  \prftree[r]{$\multimap_R$}{  \Gamma  ,  A     \vdash  B    }{ \Gamma  \vdash   A  \multimap B     }
\end{tabularx}


\subsubsection*{RuleBin}

\begin{tabularx}{\linewidth}{CCC}
  \prftree[r]{$\otimes_R$}{ \Delta  \vdash  B    }{ \Gamma  \vdash  A    }{  \Gamma  , \Delta   \vdash   A  \otimes B     }
&
  \prftree[r]{$\multimap_L$}{  \Delta  ,  B     \vdash  C    }{ \Gamma  \vdash  A    }{  (  \Gamma  , \Delta   ) ,   A  \multimap B      \vdash  C    }
\end{tabularx}



\end{multicols}
\end{document}
